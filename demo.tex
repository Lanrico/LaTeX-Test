\documentclass[utf8]{ctexart}
\usepackage{graphicx}
\usepackage{subfigure}
\usepackage{listings}
\usepackage{caption}
\usepackage{algorithm2e}
\usepackage{amsmath}
\usepackage{bm}
\usepackage{natbib}

\title{Hello, KAMI}
\author{Guanlan Ji}
\date{\today}

\begin{document}
\maketitle % 有这个才有标题和作者啥的哦

\tableofcontents

\section{一级标题}
\subsection{二级标题}
\subsection*{二级标题没有序号版}
\subsubsection{三级标题}
第一段
分段啦

第二段

第三段\par这说明\textbackslash par是可以强制分段的 %一般反斜杠需要引出其他指令,所以反斜杠是需要用\textbackslash输出的,单独的一个\加空格是空行

第四段又强制\\分段惹

\newpage % 换页
\section{一级标题}
分页惹

\section{插入图片}
\subsection{单张图片+自动生成标题}
先导包graphicx
\begin{figure}[htbp] % [htbp]的作用是自动选择插入图片的最优位置
  \centering % 设置居中
  \includegraphics[width=5cm]{testImg.jpg}
  \caption{测试图片}
\end{figure}

\subsection{两张并排图片}
\begin{figure}
  \centering
  \subfigure[第一张图]{
    \includegraphics[width=.45\textwidth]{testImg2.jpg} %宽度按百分比
  }
  \quad
  \subfigure[第二张图]{
    \includegraphics[width=.45\textwidth]{testImg2.jpg}
    % \label{测试}%文中引用该图片代号
  }
  \caption{多图示例}
\end{figure}

正文中这样引用图片

\section{列表和表格}
\subsection{列表}
\subsubsection{无序列表}
\begin{itemize}
  \item[*] *号作标记
  \item[+] +号作标记
  \item[.] .号作标记
\end{itemize}

\subsubsection{有序列表}
\begin{enumerate}
  \item[1] aabb
  \item[2] ccdd 
\end{enumerate}

\subsubsection{描述}
\begin{description}
  \item[1] 描述1
  \item[2] 描述2 
\end{description}

\subsection{表格}
\begin{table}[htbp]
  \begin{center}
  
  \setlength\tabcolsep{40pt} %调整列间距
  表1\quad符号说明
  
  \renewcommand{\arraystretch}{1.4} %调整行间距
  \begin{tabular}{c c} %表示表内元素居中(还可更换为l左r右)
  \hline %横线
  符号     & 含义                  \\ \hline
  $E_i$ & 第$i$个企业     \\  
  $r_i$      & 企业$E_i$的评价指标向量      \\  
  $w$      & 层次分析法中的权重向量        \\  
  $h_i$      & 企业$E_i$的信贷风险 \\ 
  $\alpha_i$      &企业$E_i$的年利率         \\ 
  $k$      & 不同类型的企业的受影响程度    \\ 
   \hline
  \end{tabular}
  
  \end{center}
\end{table}
\newpage % 换页

\section{代码}
\subsection{一般代码}
需要宏包listings
\lstset{
	%backgroundcolor=\color{red!50!green!50!blue!50},%代码块背景色为浅灰色
%	rulesepcolor= \color{gray}, %代码块边框颜色
	breaklines=true,  %代码过长则换行
	numbers=left, %行号在左侧显示
	numberstyle= \small,%行号字体
	% keywordstyle= \color{blue},%关键字颜色
	commentstyle=\color{gray}, %注释颜色
	frame=shadowbox    %用方框框住代码块
	frame=single,
	escapeinside=``  % 代码包含中文,把想插入的代码中的中文部分用 ` ` 括起来
}
\begin{lstlisting}[language=python]
  print('Hello,World!')
\end{lstlisting}

\subsection{伪代码}
需要宏包algorithm2e

\begin{algorithm}
  \KwData{Dataset and hyperparameters}
  Initialize $f_0(x)$;
  \For{$k = 1,2,...,n $}
  {
    Sample a set of hyperparameters from the distribution\;
    \quad params = sample\_hyperparameters(param\_distribution)\;
    Train and evaluate the model with the sampled hyperparameters\;
    \quad model.set\_params(**params)\;
    \quad scores = cross\_val\_score(model, X, y, scoring=scoring, cv=cv)\;
    \quad mean\_score = np.mean(scores)\;
    Update the best hyperparameters if the score is better:
    \If{mean\_score $\geq$ best\_score}{best\_score = mean\_score\;
            best\_params = params\;}  
    \textbf{Return:}\ best\_score, best\_params
  }
  \caption{RandomizedSearchCV algorithm}\label{Algorithm 2}
\end{algorithm}

\section{公式}
\begin{equation}\label{kami}
  F_\text{kami}=d_1\left(x^{'}(t)-X^{'}(t)\right).    
\end{equation}  
引用了公公又式式\eqref{kami},需要宏包amsmath和bm

如果$a>b$,$b>c$,则$a>c$.  
%行内公式

$$
\frac{a}{b}=a/b  
$$  
%行间公式与分式

\begin{equation}  
    a^{2}_{0}=a_{0}\times a_0=a_{0} \cdot a_{0}  
\end{equation}
%行间公式与上下标

\begin{equation*}  
\sum_{n=0}^{+\infty}\left(1+ \frac{1}{n}\right)^n  
\end{equation*}

$\overrightarrow{\bm{AB}}=\overrightarrow{AO}+\overrightarrow{OB}$
%关于向量以及公式加粗

\begin{equation*}
f(x)= \begin{cases}
x, & x>0 \\ 
-x, & x \leq 0
\end{cases}
\end{equation*}
%如何写分段函数的大括号

\begin{equation*}
\begin{aligned}
9 & =4+5 \\
& =3+6 \\
= & 1+8
\end{aligned}
\end{equation*}
%注意是在&的位置进行对齐

\begin{equation*}
\begin{bmatrix}
    11 & 4 \\
    5 & 14
\end{bmatrix}
\end{equation*}
%如何输入矩阵
\end{document}

\section{References}
需要宏包natbib

近年来,以机器学习、知识图谱为代表的人工智能技术逐渐变得普及。从车牌识别、人脸识别、语音识别、智能助手、
推荐系统到自动驾驶,人们在日常生活中都可能有意无意地用到了人工智能技术。深度学习以神经网络为主要模型,
一开始用来解决机器学习中的表示学习问题.但是由于其强大的能力,深度学习越来越多地用来解决一些通用人工智能问题,
比如推理、决策等

%以上花括号里的字母与参考文献后面的字母保持一致即可引用。

\begin{thebibliography}{99}
    % \bibitem{a}邱锡鹏. \emph{神经网络与深度学习}[M]. 机械工业出版社, 2020.
    % \bibitem{b}周志华, 王珏. \emph{机器学习及其应用}2009[M]. 清华大学出版社, 2009.
    % \bibitem{c}蒋宗礼.\emph{人工神经网络导论}[M]. 北京:高等教育出版,2008 :40-44
    % \bibitem{d}李超,范淼.\emph{python 机器学习及实践———从零开始通往Kaggle竞赛之路}[M].北京:清华大学出版社,2016:74-75
\end{thebibliography}

\end{document}